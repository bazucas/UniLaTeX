\documentclass[a4paper,12pt]{report}
\usepackage[utf8]{inputenc}
\usepackage[portuguese]{babel}
\usepackage[style=apa, backend=biber]{biblatex}
\addbibresource{biblio.bib} % Nome do arquivo de bibliografia
\usepackage{csquotes} % Necessário para citações com babel
\usepackage{graphicx}
\usepackage{amsmath}
\usepackage{hyperref}
\usepackage{geometry}
\usepackage{setspace}
\usepackage{indentfirst} % Pacote para indentar o primeiro parágrafo
\usepackage{newtxtext,newtxmath} % Usa fontes Times modernas
\usepackage{xcolor} % Pacote para cores
\usepackage{fancyhdr} % Para personalização do rodapé
\usepackage{titlesec} % Para personalizar títulos e subtítulos

% Configuração de Margens e Espaçamento
\geometry{a4paper, margin=2.5cm}
\setstretch{1.5}

% Configuração de Tamanhos de Fonte e Espaçamento para Seções
\titleformat{\section} % Títulos principais numerados
{\normalfont\fontsize{14}{17}\bfseries}{\thesection}{1em}{}
\titleformat*{\section} % Títulos principais não numerados
{\normalfont\fontsize{14}{17}\bfseries}
\titleformat{\subsection} % Subtítulos
{\normalfont\fontsize{12}{15}\bfseries}{\thesubsection}{1em}{}

% Ajuste de Espaçamento para Seções
\titlespacing{\section}
{0pt}{1.5em}{1em}
\titlespacing*{\section}
{0pt}{1.5em}{1em}
\titlespacing{\subsection}
{0pt}{1.2em}{0.8em}

% Configuração de Rodapé com Numeração
\pagestyle{fancy}
\fancyhf{} % Limpa cabeçalho e rodapé
\fancyfoot[R]{\thepage} % Numeração no canto inferior direito
\renewcommand{\headrulewidth}{0pt} % Remove linha do cabeçalho
\renewcommand{\footrulewidth}{0pt} % Remove linha do rodapé

% Configuração de Cores
\definecolor{barraazul}{RGB}{0, 51, 153}

% Numeração das seções sem números iniciais adicionais
\renewcommand{\thesection}{\arabic{section}}
\renewcommand{\thesubsection}{\arabic{section}.\arabic{subsection}}
\renewcommand{\thesubsubsection}{\arabic{section}.\arabic{subsection}.\arabic{subsubsection}}

% Redefinição do Cabeçalho da Bibliografia
\defbibheading{bibliography}{
	\section*{Referências Bibliográficas}
	\addcontentsline{toc}{section}{Referências Bibliográficas}
	\thispagestyle{fancy} % Garante que o rodapé seja consistente
}

% Garante que o estilo de página seja 'fancy' na bibliografia
\AtBeginBibliography{\pagestyle{fancy}}

% Ajuste para indentação de parágrafos
\setlength{\parindent}{1.25cm}

\begin{document}
	
	% Capa
	\begin{titlepage}
		\centering
		\vspace*{-2cm} % Reduz a margem superior da página de título
		
		% Ajusta as imagens para remover espaçamentos internos (se houver)
		\raisebox{-0.5\height}{\includegraphics[width=0.4\textwidth]{iscte.png}}%
		\hfill%
		\raisebox{-0.46\height}{\includegraphics[width=0.4\textwidth]{ista.png}}\\[0.5cm]
		
		\noindent
		{\color{barraazul}\rule{\textwidth}{1mm}} % Barra azul horizontal
		\\[1cm]
		
		{\LARGE  \textbf{Análise de Sentimentos em IA} \par}
		\vspace{1.5cm}
		
		{\Large \textbf{Mestrado em Inteligência Artificial}} \par
		\vspace{3cm}
		
		{\large \textbf{Aluno: Luís Ricardo Silva Inácio}} \par
		{\large \textbf{Aluno: João Costa}} \par
		{\large \textbf{Aluno: Diogo}} \par

		\vspace{3cm}
		
		{\large \textbf{Unidade Curricular: Cognição e Emoção}} \par
		\vspace{1cm}
		
		{\large \textbf{Professora: Doutora Cristiane da Anunciação Souza}} \par
		\vfill
		
		{\large \textbf{Data de Entrega: 29 de novembro de 2024}} \par
	\end{titlepage}
	
	
	% Costas da Capa (Em branco)
	\newpage
	\thispagestyle{empty}
	\mbox{}
	\newpage
	
	% Configuração para numeração romana
	\pagenumbering{roman}
	
	% Resumo
	\section*{Resumo}
	\addcontentsline{toc}{section}{Resumo}
	
	Este artigo analisa de forma abrangente o papel das emoções no uso de heurísticas e como a 
	
	\vspace{4em}
	
	\noindent\textbf{Palavras-Chave:} \normalsize{Emoções, }
	
	\newpage
	
	% Configuração para numeração arábica
	\pagenumbering{arabic}
	
	% Introdução
	\section{Introdução}
	
	As emoções desempenham um papel fundamental na forma como os indivíduos processam informações 
	
	\section{O Papel das Emoções nas Heurísticas e a Complexidade da Linguagem na IA Emocional}
	
	\subsection{Heurísticas, Emoções e Linguagem: Uma Introdução Teórica}
	
	\textcite{kahneman1974} definiram heurísticas como mecanismos que permitem aos indivíduos lidar com problemas de forma eficiente, utilizando uma quantidade limitada de recursos 
	
	\section{Conclusão}
	
	As emoções 
	
	\newpage
	
	% Referências
	\printbibliography
	
\end{document}
