% Dependencias
\documentclass[a4paper,12pt]{report}

% Pacotes básicos
\usepackage[portuguese]{babel} % Idioma
\usepackage{graphicx} % Gráficos e imagens
\usepackage{amsmath} % Matemática avançada
\usepackage{geometry} % Margens e layout
\usepackage{setspace} % Espaçamento entre linhas
\usepackage{titlesec} % Personalização de títulos
\usepackage{tocloft} % Índice
\usepackage{longtable} % Tabelas longas
\usepackage{lipsum} % Texto fictício
\usepackage{enumitem} % Listas personalizáveis
\usepackage{fontspec} % Fontes personalizadas (necessário com XeLaTeX ou LuaLaTeX)
\usepackage{xcolor} % Cores
\usepackage[hidelinks]{hyperref} % Hiperlinks sem caixas
\usepackage{ragged2e} % Justificação de texto
\usepackage[natbibapa]{apacite} % Estilo APA para citações
\usepackage{url} % URLs
\usepackage{indentfirst} % Força a indentação do primeiro parágrafo

% Bibliografia
\bibliographystyle{apacite}

% Configuração de margens
\geometry{a4paper, top=2cm, bottom=2.5cm, left=2.5cm, right=2.5cm}

% Espaçamento entre linhas
\setstretch{1.5}

% Fontes
\setmainfont{Times New Roman}

% Configuração de títulos
% Configuração de títulos de capítulos alinhados à esquerda com tamanho 14pt
\titleformat{\chapter}[hang]
{\normalfont\bfseries\fontsize{14pt}{16pt}\selectfont} % Formatação do texto (14pt de tamanho, 16pt de espaçamento entre linhas)
{\thechapter.}{1em} % Número do capítulo seguido por um espaço de 1em
{\vspace{-1em}} % Reduz o espaço antes do título

% Configuração de subtítulos (sections) com tamanho 12pt
\titleformat{\section}[block]
{\normalfont\bfseries\fontsize{12pt}{14pt}\selectfont} % Formatação do texto (12pt de tamanho, 14pt de espaçamento entre linhas)
{\thesection}{1em}{}

% Configuração de títulos do Índice, Lista de Figuras e Lista de Tabelas
% Redefinir os títulos manualmente para o mesmo tamanho de fonte (14pt)
\addto\captionsportuguese{
	\renewcommand{\contentsname}{\fontsize{14pt}{16pt}\selectfont Índice}
	\renewcommand{\listfigurename}{\fontsize{14pt}{16pt}\selectfont Lista de Figuras}
	\renewcommand{\listtablename}{\fontsize{14pt}{16pt}\selectfont Lista de Tabelas}
}

% Ajuste de cabeçalhos
\usepackage{fancyhdr}
\fancypagestyle{conteudo}{
	\fancyhf{} % Limpa cabeçalhos e rodapés
	\fancyhead[R]{\small \nouppercase{\leftmark}} % Título do capítulo no canto superior direito
	\renewcommand{\headrulewidth}{0pt} % Remove a linha no cabeçalho
}

% Reduz o espaçamento entre linhas dos índices
\usepackage{etoolbox}
\patchcmd{\tableofcontents}{\@starttoc{toc}}{\@starttoc{toc}\setlength{\parskip}{-5pt}}{}{}
\patchcmd{\listoffigures}{\@starttoc{lof}}{\@starttoc{lof}\setlength{\parskip}{-5pt}}{}{}
\patchcmd{\listoftables}{\@starttoc{lot}}{\@starttoc{lot}\setlength{\parskip}{-5pt}}{}{}

% Configuração de cores
\definecolor{barraazul}{RGB}{0, 51, 153}

% Corrigir problemas de viúvas e órfãs
\clubpenalty=10000
\widowpenalty=10000
\sloppy


\begin{document}
	
	% Capa
	\input{capa}
	
	% Página em Branco
	% Página em Branco
\newpage
\thispagestyle{empty}
\mbox{}
\newpage
	
	% Contracapa
	\begin{titlepage}
	\begin{flushleft}
		\begin{minipage}{\textwidth}
			\vspace{1cm} % Espaçamento antes da imagem
			\includegraphics[width=0.5\textwidth]{ista.png}
		\end{minipage}\\[1cm]
	\end{flushleft}
	\noindent
	\textcolor{barraazul}{\rule{\textwidth}{1mm}} % Barra azul
	\\[0.5cm]
	{\Large \textbf{\centering Departamento de Ciências e Tecnologias da Informação}}\\[1cm]
	{\Huge \textbf{\centering Desenvolvimento de modelos preditivos com base em RNN}}\\[1.5cm]
	\noindent
	\textbf{Luís Ricardo Silva Inácio}\\
	\textbf{Número de Aluno: 129074}\\[2cm]
	\textbf{Mestrado em Inteligência Artificial}\\[1.5cm]
	\textbf{Orientador: Tozé Brito, Phd}\\
	\textbf{Coorientador: Rui Brito, Phd}\\[3cm]
	\textbf{Outubro, 2024}
\end{titlepage}

	
	% Página de Copyright
	\newpage
\thispagestyle{empty}
\vspace*{1,5cm} % Adiciona uma margem superior fixa de 2 cm
\noindent % Evita indentação na primeira linha
{\footnotesize % Tamanho menor de texto (menor que \small)
	\textbf{Direitos de cópia ou Copyright} \\ % Título
	\textcopyright
	Copyright: Luís Ricardo Silva Inácio \\[-0.5cm] % Reduz o espaço após o autor
	\begin{flushleft} % Início do ambiente para alinhar à esquerda e justificar
		\renewcommand{\baselinestretch}{1}\selectfont % Define o espaçamento entre linhas
		\justify % Garante a justificação do texto
		O Iscte - Instituto Universitário de Lisboa tem o direito, perpétuo e sem limites geográficos, de arquivar e publicitar este trabalho através de exemplares impressos reproduzidos em papel ou de forma digital, ou por qualquer outro meio conhecido ou que venha a ser inventado, de o divulgar através de repositórios científicos e de admitir a sua cópia e distribuição com objetivos educacionais ou de investigação, não comerciais, desde que seja dado crédito ao autor e editor.
	\end{flushleft}
}
\newpage

	
	% Página de Agradecimentos
		% Página de Agradecimentos
\chapter*{Agradecimentos}
\addcontentsline{toc}{chapter}{Agradecimentos}
Gostaria de expressar a minha gratidão a todas as pessoas que me apoiaram durante a realização deste trabalho...
	
	% Resumo
	% Resumo
\newpage
\chapter*{Resumo}
\addcontentsline{toc}{chapter}{Resumo}
Texto do resumo em português. \\[1em]
\textbf{Palavras-chave:} palavra-chave1, palavra-chave2, palavra-chave3.
\newpage
	
	% Abstract
	% Abstract
\chapter*{Abstract}
\addcontentsline{toc}{chapter}{Abstract}
Texto do resumo em inglês. \\[1em]
\textbf{Keywords:} keyword1, keyword2, keyword3.
\newpage
	
	% Índices
	\tableofcontents
	\newpage
	\listoffigures
	\newpage
	\listoftables
	\newpage
	
	% Lista de Abreviaturas e Siglas (com glossário)
	\chapter*{Lista de Abreviaturas e Siglas}
	\addcontentsline{toc}{chapter}{Lista de Abreviaturas e Siglas}
	\begin{longtable}{p{3cm}p{10cm}}
		\textbf{Sigla} & \textbf{Descrição} \\
		\hline
		API & Application Programming Interface \\
		BI & Business Intelligence \\
		KPI & Key Performance Indicator \\
		Deep Learning & Subcampo do machine learning que utiliza redes neurais profundas. \\
		Hiperparâmetros & Parâmetros ajustados antes do treinamento do modelo. \\
		Overfitting & Quando o modelo se ajusta demais aos dados de treinamento. \\
		Batch Size & Número de exemplos processados por vez durante o treinamento. \\
	\end{longtable}
	
	% Começar numeração árabe
	\newpage
	\pagenumbering{arabic}
	
	% Ativar cabeçalho personalizado
	\pagestyle{conteudo}
	
	% Capítulos
	\chapter{Introdução}
	\section{Definição e Contexto}
	A inteligência artificial tem evoluído significativamente nas últimas décadas, com avanços em redes neuronais profundas sendo especialmente notáveis. O livro seminal de \citet{goodfellow2016} destaca como o deep learning transformou o campo.
	
	\begin{figure}[ht]
		\centering
		\includegraphics[width=0.5\textwidth]{example-image-a}
		\caption{Esquema de aprendizado em redes neurais.}
		\label{fig:esquema-aprendizado}
	\end{figure}
	
	\section{Desafios Atuais}
	Os desafios associados à implementação de redes neuronais foram explorados em várias pesquisas. Por exemplo, \citet{rao2019} argumenta que os principais desafios incluem o ajuste de hiperparâmetros e a escalabilidade.
	
	\begin{table}[ht]
		\centering
		\begin{tabular}{|l|l|l|}
			\hline
			\textbf{Parâmetro} & \textbf{Descrição} & \textbf{Valor} \\ \hline
			Taxa de Aprendizagem & Controla o ajuste do modelo & 0.001 \\ \hline
			Número de Camadas & Define a profundidade do modelo & 4 \\ \hline
			Tamanho do Batch & Exemplos por iteração & 32 \\ \hline
		\end{tabular}
		\caption{Parâmetros comuns em redes neurais.}
		\label{tab:parametros-redes}
	\end{table}
	
	\chapter{Revisão de Literatura}
	\section{Avanços em Otimização}
	A literatura recente investigou estratégias para otimizar redes neuronais e melhorar a eficiência. \citet{smith2021optimization} analisaram técnicas avançadas de otimização que têm impacto no desempenho.
	
	\begin{figure}[ht]
		\centering
		\includegraphics[width=0.7\textwidth]{example-image-b}
		\caption{Gráfico de convergência em redes treinadas.}
		\label{fig:convergencia}
	\end{figure}
	
	\section{Aplicações de Machine Learning}
	\citet{brown2020} exploraram o papel do aprendizado de máquina na análise preditiva, destacando sua relevância em saúde e finanças.
	
	\begin{table}[ht]
		\centering
		\begin{tabular}{|l|l|}
			\hline
			\textbf{Setor} & \textbf{Exemplo} \\ \hline
			Saúde & Diagnóstico médico \\ \hline
			Finanças & Previsão de fraudes \\ \hline
		\end{tabular}
		\caption{Exemplos de aplicações de machine learning.}
		\label{tab:aplicacoes}
	\end{table}
	
	\chapter{Metodologia}
	\section{Abordagens de Treinamento}
	A metodologia deste estudo foi baseada em \citet{smith2021optimization}, que enfatizam o uso de técnicas otimizadas para treinar redes profundas.
	
	\begin{figure}[ht]
		\centering
		\includegraphics[width=1.0\textwidth]{example-image-c}
		\caption{Diagrama detalhado de uma rede convolucional.}
		\label{fig:rede-convolucional}
	\end{figure}
	
	\chapter{Resultados}
	\section{Melhorias Observadas}
	Os resultados corroboram os achados de \citet{smith2021optimization}, demonstrando melhorias significativas no desempenho.
	
	% Desativar cabeçalho personalizado antes das referências
	\clearpage
	\pagestyle{plain}
	
	% Referências Bibliográficas
	\addcontentsline{toc}{chapter}{Referências Bibliográficas}
	\bibliography{referencias}
	
	% Apêndices
	\appendix
	\chapter{Anexo A}
	Texto fictício para apêndice.
	\lipsum[10]
	
\end{document}
