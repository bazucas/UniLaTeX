% Dependencias
\documentclass[a4paper,12pt]{report}

% Pacotes básicos
\usepackage[portuguese]{babel} % Idioma
\usepackage{graphicx} % Gráficos e imagens
\usepackage{amsmath} % Matemática avançada
\usepackage{geometry} % Margens e layout
\usepackage{setspace} % Espaçamento entre linhas
\usepackage{titlesec} % Personalização de títulos
\usepackage{tocloft} % Índice
\usepackage{longtable} % Tabelas longas
\usepackage{lipsum} % Texto fictício
\usepackage{enumitem} % Listas personalizáveis
\usepackage{fontspec} % Fontes personalizadas (necessário com XeLaTeX ou LuaLaTeX)
\usepackage{xcolor} % Cores
\usepackage[hidelinks]{hyperref} % Hiperlinks sem caixas
\usepackage{ragged2e} % Justificação de texto
\usepackage[natbibapa]{apacite} % Estilo APA para citações
\usepackage{url} % URLs
\usepackage{indentfirst} % Força a indentação do primeiro parágrafo

% Bibliografia
\bibliographystyle{apacite}

% Configuração de margens
\geometry{a4paper, top=2cm, bottom=2.5cm, left=2.5cm, right=2.5cm}

% Espaçamento entre linhas
\setstretch{1.5}

% Fontes
\setmainfont{Times New Roman}

% Configuração de títulos
% Configuração de títulos de capítulos alinhados à esquerda com tamanho 14pt
\titleformat{\chapter}[hang]
{\normalfont\bfseries\fontsize{14pt}{16pt}\selectfont} % Formatação do texto (14pt de tamanho, 16pt de espaçamento entre linhas)
{\thechapter.}{1em} % Número do capítulo seguido por um espaço de 1em
{\vspace{-1em}} % Reduz o espaço antes do título

% Configuração de subtítulos (sections) com tamanho 12pt
\titleformat{\section}[block]
{\normalfont\bfseries\fontsize{12pt}{14pt}\selectfont} % Formatação do texto (12pt de tamanho, 14pt de espaçamento entre linhas)
{\thesection}{1em}{}

% Configuração de títulos do Índice, Lista de Figuras e Lista de Tabelas
% Redefinir os títulos manualmente para o mesmo tamanho de fonte (14pt)
\addto\captionsportuguese{
	\renewcommand{\contentsname}{\fontsize{14pt}{16pt}\selectfont Índice}
	\renewcommand{\listfigurename}{\fontsize{14pt}{16pt}\selectfont Lista de Figuras}
	\renewcommand{\listtablename}{\fontsize{14pt}{16pt}\selectfont Lista de Tabelas}
}

% Ajuste de cabeçalhos
\usepackage{fancyhdr}
\fancypagestyle{conteudo}{
	\fancyhf{} % Limpa cabeçalhos e rodapés
	\fancyhead[R]{\small \nouppercase{\leftmark}} % Título do capítulo no canto superior direito
	\renewcommand{\headrulewidth}{0pt} % Remove a linha no cabeçalho
}

% Reduz o espaçamento entre linhas dos índices
\usepackage{etoolbox}
\patchcmd{\tableofcontents}{\@starttoc{toc}}{\@starttoc{toc}\setlength{\parskip}{-5pt}}{}{}
\patchcmd{\listoffigures}{\@starttoc{lof}}{\@starttoc{lof}\setlength{\parskip}{-5pt}}{}{}
\patchcmd{\listoftables}{\@starttoc{lot}}{\@starttoc{lot}\setlength{\parskip}{-5pt}}{}{}

% Configuração de cores
\definecolor{barraazul}{RGB}{0, 51, 153}

% Corrigir problemas de viúvas e órfãs
\clubpenalty=10000
\widowpenalty=10000
\sloppy


\begin{document}
	
	% Capa
	\input{capa}
	
	% Página em Branco
	% Página em Branco
\newpage
\thispagestyle{empty}
\mbox{}
\newpage
	
	% Contracapa
	\begin{titlepage}
	\begin{flushleft}
		\begin{minipage}{\textwidth}
			\vspace{1cm} % Espaçamento antes da imagem
			\includegraphics[width=0.5\textwidth]{ista.png}
		\end{minipage}\\[1cm]
	\end{flushleft}
	\noindent
	\textcolor{barraazul}{\rule{\textwidth}{1mm}} % Barra azul
	\\[0.5cm]
	{\Large \textbf{\centering Departamento de Ciências e Tecnologias da Informação}}\\[1cm]
	{\Huge \textbf{\centering Desenvolvimento de modelos preditivos com base em RNN}}\\[1.5cm]
	\noindent
	\textbf{Luís Ricardo Silva Inácio}\\
	\textbf{Número de Aluno: 129074}\\[2cm]
	\textbf{Mestrado em Inteligência Artificial}\\[1.5cm]
	\textbf{Orientador: Tozé Brito, Phd}\\
	\textbf{Coorientador: Rui Brito, Phd}\\[3cm]
	\textbf{Outubro, 2024}
\end{titlepage}

	
	% Página de Copyright
	\newpage
\thispagestyle{empty}
\vspace*{1,5cm} % Adiciona uma margem superior fixa de 2 cm
\noindent % Evita indentação na primeira linha
{\footnotesize % Tamanho menor de texto (menor que \small)
	\textbf{Direitos de cópia ou Copyright} \\ % Título
	\textcopyright
	Copyright: Luís Ricardo Silva Inácio \\[-0.5cm] % Reduz o espaço após o autor
	\begin{flushleft} % Início do ambiente para alinhar à esquerda e justificar
		\renewcommand{\baselinestretch}{1}\selectfont % Define o espaçamento entre linhas
		\justify % Garante a justificação do texto
		O Iscte - Instituto Universitário de Lisboa tem o direito, perpétuo e sem limites geográficos, de arquivar e publicitar este trabalho através de exemplares impressos reproduzidos em papel ou de forma digital, ou por qualquer outro meio conhecido ou que venha a ser inventado, de o divulgar através de repositórios científicos e de admitir a sua cópia e distribuição com objetivos educacionais ou de investigação, não comerciais, desde que seja dado crédito ao autor e editor.
	\end{flushleft}
}
\newpage

	
	% Página de Agradecimentos
		% Página de Agradecimentos
\chapter*{Agradecimentos}
\addcontentsline{toc}{chapter}{Agradecimentos}
Gostaria de expressar a minha gratidão a todas as pessoas que me apoiaram durante a realização deste trabalho...
	
	% Resumo
	% Resumo
\newpage
\chapter*{Resumo}
\addcontentsline{toc}{chapter}{Resumo}
Texto do resumo em português. \\[1em]
\textbf{Palavras-chave:} palavra-chave1, palavra-chave2, palavra-chave3.
\newpage
	
	% Abstract
	% Abstract
\chapter*{Abstract}
\addcontentsline{toc}{chapter}{Abstract}
Texto do resumo em inglês. \\[1em]
\textbf{Keywords:} keyword1, keyword2, keyword3.
\newpage
	
	% Índices
	\tableofcontents
	\newpage
	\listoffigures
	\newpage
	\listoftables
	\newpage
	
	% Lista de Abreviaturas e Siglas
	% Lista de Abreviaturas e Siglas
\chapter*{Lista de Abreviaturas e Siglas}
\addcontentsline{toc}{chapter}{Lista de Abreviaturas e Siglas}
\begin{longtable}{p{3cm} p{10cm}}
	\textbf{Sigla} & \textbf{Descrição} \\
	\hline
	API & Application Programming Interface \\
	BI & Business Intelligence \\
	KPI & Key Performance Indicator \\
\end{longtable}

	
	% Começar numeração árabe
	\newpage
	\pagenumbering{arabic}
	
	% Capítulos
	\chapter{Introdução}
	A inteligência artificial tem evoluído significativamente nas últimas décadas, com avanços em redes neuronais profundas sendo especialmente notáveis. O livro seminal de \citet{goodfellow2016} destaca como o deep learning transformou o campo, permitindo o desenvolvimento de modelos complexos para problemas de visão computacional, linguagem natural e mais. Além disso, \citet{taylor2015} enfatiza que a simplicidade de certas abordagens pode ser essencial para iniciantes entenderem os fundamentos da aprendizagem automática. 
	
	Os desafios associados à implementação e otimização de redes neuronais foram explorados em várias pesquisas. Por exemplo, \citet{rao2019} argumenta que os principais desafios incluem o ajuste de hiperparâmetros e a escalabilidade dos modelos.
	
	\chapter{Revisão de Literatura}
	A literatura recente tem investigado estratégias para otimizar redes neuronais e melhorar a eficiência dos modelos. \citet{smith2021optimization} analisaram técnicas avançadas de otimização que têm um impacto direto no desempenho de redes neuronais em tarefas críticas. Da mesma forma, \citet{brown2020} exploraram o papel da aprendizagem automática na análise preditiva, destacando a importância do machine learning para setores como saúde e finanças.
	
	Um avanço importante foi a introdução de redes convolucionais para classificação de imagens por \citet{krizhevsky2012}, que revolucionou a área com a sua abordagem baseada no conjunto de dados ImageNet. Estudos subsequentes, como os apresentados em \citet{noauthor2021}, detalham os avanços em redes recorrentes, ampliando sua aplicabilidade em processamento de séries temporais e geração de texto.
	
	Contribuições teóricas também foram fundamentais. O livro \citet{mitpress2018} fornece uma base sólida sobre os princípios de aprendizagem profunda, enquanto \citet{wikipedia2024} oferece uma visão geral acessível das redes neuronais recorrentes.
	
	\chapter{Metodologia}
	A metodologia deste estudo foi baseada em abordagens sugeridas por \citet{smith2021optimization}, que enfatizam a utilização de técnicas otimizadas para treinar redes profundas. Além disso, os parâmetros do modelo foram ajustados com base nos princípios descritos por \citet{rao2019}, garantindo um equilíbrio entre desempenho e complexidade computacional.
	
	O framework experimental foi inspirado nas estratégias utilizadas em \citet{krizhevsky2012}, adaptando redes convolucionais para novos conjuntos de dados. Também foram incorporadas técnicas de análise preditiva baseadas nas metodologias descritas por \citet{brown2020}.
	
	\chapter{Resultados}
	Os resultados obtidos corroboram os achados de \citet{smith2021optimization}, demonstrando melhorias significativas na eficiência do modelo ao adotar técnicas avançadas de otimização. Adicionalmente, os modelos baseados em redes convolucionais apresentaram precisão semelhante às descritas por \citet{krizhevsky2012}, validando sua robustez em tarefas de classificação de imagens.
	
	Curiosamente, as limitações de redes recorrentes, como discutido por \citet{noauthor2021}, foram observadas em tarefas de processamento sequencial, reforçando a necessidade de arquiteturas híbridas para superar esses desafios.
	
	\chapter{Conclusão}
	Com base nas evidências apresentadas, pode-se concluir que a adoção de técnicas modernas de otimização e arquiteturas avançadas de redes neuronais desempenha um papel crucial no avanço da inteligência artificial. Conforme discutido em \citet{goodfellow2016}, o futuro do deep learning depende da contínua integração de métodos teóricos e práticos.
	
	Além disso, as contribuições de \citet{rao2019} e \citet{brown2020} destacam que uma abordagem multidisciplinar é essencial para enfrentar os desafios associados à escalabilidade e aplicação prática dos modelos. Este trabalho, portanto, reforça a importância da pesquisa colaborativa e interdisciplinar no progresso da inteligência artificial.
	
	% Referências Bibliográficas
	\addcontentsline{toc}{chapter}{Referências Bibliográficas}
	\bibliography{referencias}
	
	% Apêndices
	\appendix
	\chapter{Anexo A}
	Texto fictício para apêndice.
	\lipsum[10]
	
\end{document}
