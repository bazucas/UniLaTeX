\section{Drawbacks of an electoral democracy}\label{sec:benefits_drawbacks}

    % https://www.youtube.com/watch?v=wKLKNJbCNuM

    This section explores the benefits and drawbacks of each component within the integrated infrastructure stack, comprising OutSystems, Azure, and MongoDB.
    
    \subsection{MongoDB}
    
    \subsubsection{Advantages}
    MongoDB, as a NoSQL database, offers several advantages in the context of this infrastructure stack:
    
    \begin{itemize}
        \item **Scalability:** MongoDB's flexible architecture allows for seamless horizontal scaling, accommodating the demands of large-scale applications.
        \item **Schema Flexibility:** The document-oriented model provides schema flexibility, enabling easy adaptation to evolving application requirements.
        \item **High Performance:** MongoDB's efficient indexing and querying mechanisms contribute to high performance, crucial for data-intensive applications.
        \item **JSON-Like Documents:** Storing data in JSON-like BSON format enhances data representation and supports complex structures.
    \end{itemize}
    
    \subsubsection{Drawbacks}
    Despite its strengths, MongoDB has certain limitations:
    
    \begin{itemize}
        \item **Join Operations:** MongoDB lacks native support for join operations, which may pose challenges for complex data relationships.
        \item **Transactions:** While recent versions introduced transaction support, MongoDB historically had limitations in supporting multi-document transactions.
    \end{itemize}
    
    \subsection{Azure}
    
    \subsubsection{Advantages}
    Azure, Microsoft's cloud platform, offers numerous advantages in this infrastructure stack:
    
    \begin{itemize}
        \item **Scalability and Flexibility:** Azure provides scalable and flexible cloud services, allowing applications to adapt to changing workloads.
        \item **Integration Capabilities:** Seamless integration with other Microsoft products and services enhances overall ecosystem cohesion.
        \item **Security:** Azure implements robust security measures, ensuring the confidentiality and integrity of stored and processed data.
        \item **Global Reach:** With data centers worldwide, Azure facilitates global deployment, reducing latency and improving user experience.
    \end{itemize}
    
    \subsubsection{Drawbacks}
    However, Azure presents some challenges:
    
    \begin{itemize}
        \item **Cost Complexity:** The diverse range of services may result in cost complexity, requiring careful planning to optimize expenses.
        \item **Learning Curve:** Mastering the full spectrum of Azure services may have a steep learning curve for development teams.
    \end{itemize}
    
    \subsection{OutSystems}
    
    \subsubsection{Advantages}
    OutSystems, as a low-code platform, offers distinctive advantages:
    
    \begin{itemize}
        \item **Rapid Development:** The visual development environment accelerates application development, reducing time-to-market.
        \item **Cross-Platform Compatibility:** OutSystems supports cross-platform development, enhancing the reach of applications.
        \item **Integration Capabilities:** Robust integration features simplify the incorporation of third-party services and databases.
        \item **Maintainability:** The low-code paradigm enhances application maintainability, fostering collaboration between development and business teams.
    \end{itemize}
    
    \subsubsection{Drawbacks}
    However, there are considerations to bear in mind:
    
    \begin{itemize}
        \item **Customization Limitations:** Highly customized and complex applications may face limitations within the low-code paradigm.
        \item **Vendor Lock-in:** Depending extensively on a low-code platform may lead to vendor lock-in concerns.
    \end{itemize}
    
    In conclusion, understanding the nuanced benefits and drawbacks of each component in the infrastructure stack is essential for making informed decisions during the development and deployment phases.