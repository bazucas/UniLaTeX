\documentclass[a4paper,12pt]{report}
\usepackage[utf8]{inputenc}
\usepackage[portuguese]{babel}
\usepackage{graphicx}
\usepackage{amsmath}
\usepackage{geometry}
\usepackage{setspace}
\usepackage{titlesec}
\usepackage{tocloft}
\usepackage{longtable}
\usepackage{lipsum}
\usepackage{enumitem}
\usepackage{fontspec}
\usepackage{xcolor}
\usepackage[hidelinks]{hyperref}
\usepackage{ragged2e}
\usepackage[natbibapa]{apacite} % Estilo APA para o natbib
\usepackage{natbib} % Necessário para \citep e \citet
\usepackage{url}

% Configuração da Margem e Espaçamento
\geometry{a4paper, margin=2.5cm}
\setstretch{1.5}

% Configuração de Fontes
\setmainfont{Times New Roman}

% Configuração de Títulos
\titleformat{\chapter}[block]{\normalfont\huge\bfseries}{\thechapter.}{20pt}{\centering}
\titleformat{\section}[block]{\normalfont\Large\bfseries}{\thesection}{1em}{}

% Configuração do Índice
\renewcommand{\contentsname}{Índice} % Define o título como "Índice"
\renewcommand{\cftchapfont}{\bfseries}
\renewcommand{\cftsecfont}{\normalfont}
\renewcommand{\cftsubsecfont}{\itshape}
\renewcommand{\cftchappagefont}{\bfseries}
\renewcommand{\cftsecleader}{\cftdotfill{\cftdotsep}} % Ativa os pontos ligando títulos às páginas

% Configuração de Cores
\definecolor{barraazul}{RGB}{0, 51, 153}

% Começar numeração romana
\pagenumbering{roman}

\begin{document}
	
	% Capa
	\begin{titlepage}
		\begin{flushleft}
			\includegraphics[width=0.5\textwidth]{iscte.png}\\[1cm]
		\end{flushleft}
		\noindent
		\textcolor{barraazul}{\rule{\textwidth}{1mm}} % Barra azul
		\\[0.5cm]
		{\Huge \textbf{\centering Desenvolvimento de modelos preditivos com base em RNN}}\\[1.5cm]
		\noindent
		\textbf{Luís Ricardo Silva Inácio}\\
		\textbf{Número de Aluno: 129074}\\[2cm]
		\textbf{Mestrado em Inteligência Artificial}\\[1.5cm]
		\textbf{Orientador: Tozé Brito, Phd}\\
		\textbf{Coorientador: (se aplicável)}\\[3cm]
		\textbf{Outubro, 2024}
	\end{titlepage}
	
	% Página em Branco
	\newpage
	\thispagestyle{empty}
	\mbox{}
	\newpage
	
	% Contracapa
	\begin{titlepage}
		\begin{flushleft}
			\includegraphics[width=0.5\textwidth]{ista.png}\\[1cm]
		\end{flushleft}
		\noindent
		\textcolor{barraazul}{\rule{\textwidth}{1mm}} % Barra azul
		\\[0.5cm]
		{\Large \textbf{\centering Departamento de Ciências e Tecnologias da Informação}}\\[1cm]
		{\Huge \textbf{\centering Desenvolvimento de modelos preditivos com base em RNN}}\\[1.5cm]
		\noindent
		\textbf{Luís Ricardo Silva Inácio}\\
		\textbf{Número de Aluno: 129074}\\[2cm]
		\textbf{Orientador: Tozé Brito, Phd}\\
		\textbf{Coorientador: (se aplicável)}\\[3cm]
		\textbf{Outubro, 2024}
	\end{titlepage}
	
	% Página de Copyright
	\newpage
	\thispagestyle{empty}
	\noindent % Evita indentação na primeira linha
	{\footnotesize % Tamanho menor de texto (menor que \small)
		\textbf{Direitos de cópia ou Copyright} \\ % Título
		\textcopyright
		Copyright: Luís Ricardo Silva Inácio \\[-0.5cm] % Reduz o espaço após o autor
		\begin{flushleft} % Início do ambiente para alinhar à esquerda e justificar
			\renewcommand{\baselinestretch}{1}\selectfont % Define o espaçamento entre linhas
			\justify % Garante a justificação do texto
			O Iscte - Instituto Universitário de Lisboa tem o direito, perpétuo e sem limites geográficos, de arquivar e publicitar este trabalho através de exemplares impressos reproduzidos em papel ou de forma digital, ou por qualquer outro meio conhecido ou que venha a ser inventado, de o divulgar através de repositórios científicos e de admitir a sua cópia e distribuição com objetivos educacionais ou de investigação, não comerciais, desde que seja dado crédito ao autor e editor.
		\end{flushleft}
	}
	\newpage
	
	% Página de Agradecimentos
	\chapter*{Agradecimentos}
	\addcontentsline{toc}{chapter}{Agradecimentos}
	Gostaria de expressar a minha gratidão a todas as pessoas que me apoiaram durante a realização deste trabalho...
	
	% Resumo
	\newpage
	\chapter*{Resumo}
	\addcontentsline{toc}{chapter}{Resumo}
	Texto do resumo em português. \\[1em]
	\textbf{Palavras-chave:} Redes Neurais Recorrentes, Modelos Preditivos, Machine Learning.
	\newpage
	
	% Abstract
	\chapter*{Abstract}
	\addcontentsline{toc}{chapter}{Abstract}
	Texto do resumo em inglês. \\[1em]
	\textbf{Keywords:} Recurrent Neural Networks, Predictive Models, Machine Learning.
	\newpage
	
	% Índices
	\tableofcontents
	\newpage
	\listoffigures
	\newpage
	\listoftables
	\newpage
	
	% Lista de Abreviaturas e Siglas
	\chapter*{Lista de Abreviaturas e Siglas}
	\addcontentsline{toc}{chapter}{Lista de Abreviaturas e Siglas}
	\begin{longtable}{p{3cm} p{10cm}}
		\textbf{Sigla} & \textbf{Descrição} \\
		\hline
		RNN & Recurrent Neural Network \\
		AI & Artificial Intelligence \\
		ML & Machine Learning \\
	\end{longtable}
	
	% Começar numeração árabe
	\newpage
	\pagenumbering{arabic}
	
	% Introdução
	\chapter{Introdução}
	A importância das Redes Neurais Recorrentes (RNNs) tem sido amplamente discutida na literatura \citep{goodfellow2016}. Este estudo tem como objetivo principal explorar...
	
	% Revisão de Literatura
	\chapter{Revisão de Literatura}
	A abordagem metodológica sugerida por Smith et al. (2021) indica que os modelos de redes neurais podem ser otimizados através de técnicas específicas \citep{smith2021optimization}. Além disso, a aplicação de redes convolucionais apresenta potencialidades adicionais \citep{krizhevsky2012}.
	
	% Metodologia
	\chapter{Metodologia}
	Para validar os modelos propostos, utilizou-se a base de dados descrita em estudos anteriores \citep{brown2020}. A abordagem experimental foi estruturada com base nas diretrizes de Goodfellow et al. \citep{goodfellow2016}.
	
	% Resultados
	\chapter{Resultados}
	Os resultados obtidos refletem a eficácia do modelo preditivo \citep{noauthor2021}, com um incremento significativo na precisão, conforme reportado na literatura secundária \citep{johnson1999as}.
	
	% Conclusão
	\chapter{Conclusão}
	O trabalho realizado destaca as vantagens das RNNs para cenários preditivos. Estudos futuros poderão focar-se em desafios apontados por Brown (2020) \citep{brown2020}.
	
	% Referências
	\chapter*{Referências}
	\addcontentsline{toc}{chapter}{Referências}
	\bibliographystyle{apacite}
	\bibliography{referencias}
	
	% Apêndices
	\appendix
	\chapter{Apêndice A}
	Texto fictício para apêndice. 
	\lipsum[10]
	
\end{document}
