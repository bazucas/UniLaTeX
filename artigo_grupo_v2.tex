\documentclass[a4paper,12pt]{report}
\usepackage[utf8]{inputenc}
\usepackage[portuguese]{babel}
\usepackage[style=apa, backend=biber]{biblatex}
\addbibresource{refs.bib} % Nome do arquivo de bibliografia
\usepackage{csquotes} % Necessário para citações com babel
\usepackage{graphicx}
\usepackage{amsmath}
\usepackage{hyperref}
\usepackage{geometry}
\usepackage{setspace}
\usepackage{indentfirst} % Pacote para indentar o primeiro parágrafo
\usepackage{newtxtext,newtxmath} % Usa fontes Times modernas
\usepackage{xcolor} % Pacote para cores
\usepackage{fancyhdr} % Para personalização do rodapé
\usepackage{titlesec} % Para personalizar títulos e subtítulos

% Configuração de Margens e Espaçamento
\geometry{a4paper, margin=2.5cm}
\setstretch{1.5}

% Configuração de Tamanhos de Fonte e Espaçamento para Seções
\titleformat{\section} % Títulos principais numerados
{\normalfont\fontsize{14}{17}\bfseries}{\thesection}{1em}{}
\titleformat*{\section} % Títulos principais não numerados
{\normalfont\fontsize{14}{17}\bfseries}
\titleformat{\subsection} % Subtítulos
{\normalfont\fontsize{12}{15}\bfseries}{\thesubsection}{1em}{}

% Ajuste de Espaçamento para Seções
\titlespacing{\section}
{0pt}{1.5em}{1em}
\titlespacing*{\section}
{0pt}{1.5em}{1em}
\titlespacing{\subsection}
{0pt}{1.2em}{0.8em}

% Configuração de Rodapé com Numeração
\pagestyle{fancy}
\fancyhf{} % Limpa cabeçalho e rodapé
\fancyfoot[R]{\thepage} % Numeração no canto inferior direito
\renewcommand{\headrulewidth}{0pt} % Remove linha do cabeçalho
\renewcommand{\footrulewidth}{0pt} % Remove linha do rodapé

% Configuração de Cores
\definecolor{barraazul}{RGB}{0, 51, 153}

% Numeração das seções sem números iniciais adicionais
\renewcommand{\thesection}{\arabic{section}}
\renewcommand{\thesubsection}{\arabic{section}.\arabic{subsection}}
\renewcommand{\thesubsubsection}{\arabic{section}.\arabic{subsection}.\arabic{subsubsection}}

% Redefinição do Cabeçalho da Bibliografia
\defbibheading{bibliography}{
	\section*{Referências Bibliográficas}
	\addcontentsline{toc}{section}{Referências Bibliográficas}
	\thispagestyle{fancy} % Garante que o rodapé seja consistente
}

% Garante que o estilo de página seja 'fancy' na bibliografia
\AtBeginBibliography{\pagestyle{fancy}}

% Ajuste para indentação de parágrafos
\setlength{\parindent}{1.25cm}

\begin{document}
	
	% Capa
	\begin{titlepage}
		\centering
		\vspace*{-2cm} % Reduz a margem superior da página de título
		
		% Ajusta as imagens para remover espaçamentos internos (se houver)
		\raisebox{-0.5\height}{\includegraphics[width=0.4\textwidth]{iscte.png}}%
		\hfill%
		\raisebox{-0.46\height}{\includegraphics[width=0.4\textwidth]{ista.png}}\\[0.5cm]
		
		\noindent
		{\color{barraazul}\rule{\textwidth}{1mm}} % Barra azul horizontal
		\\[1cm]
		
		{\LARGE  \textbf{Análise de Sentimentos em IA} \par}
		\vspace{1.5cm}
		
		{\Large \textbf{Mestrado em Inteligência Artificial}} \par
		\vspace{3cm}
		
		{\large \textbf{Aluno: Luís Ricardo Silva Inácio}} \par
		{\large \textbf{Aluno: João Costa}} \par
		{\large \textbf{Aluno: Diogo}} \par
		
		\vspace{3cm}
		
		{\large \textbf{Unidade Curricular: Cognição e Emoção}} \par
		\vspace{1cm}
		
		{\large \textbf{Professora: Doutora Cristiane da Anunciação Souza}} \par
		\vfill
		
		{\large \textbf{Data de Entrega: 29 de novembro de 2024}} \par
	\end{titlepage}
	
	
	% Costas da Capa (Em branco)
	\newpage
	\thispagestyle{empty}
	\mbox{}
	\newpage
	
	% Configuração para numeração romana
	\pagenumbering{roman}
	
	% Resumo
	\section*{Resumo}
	\addcontentsline{toc}{section}{Resumo}
	
	Este relatório analisa criticamente o desenvolvimento de modelos de IA emocional, enfocando a personalização e a análise multimodal como abordagens inovadoras. Com base em estudos recentes, discute-se a eficácia dessas metodologias na melhoria da interação humano-máquina e suas implicações éticas e sociais. O relatório também identifica pontos fortes e fragilidades dos estudos analisados, propondo caminhos futuros para a pesquisa na área.
	
	\vspace{4em}
	
	\noindent\textbf{Palavras-Chave:} \normalsize{Emoções, Inteligência Artificial, Análise Multimodal, Personalização de Modelos, Ética em IA}
	
	\newpage
	
	% Configuração para numeração arábica
	\pagenumbering{arabic}
	
	% Introdução
	\section{Introdução}
	
	O desenvolvimento de sistemas de Inteligência Artificial (IA) emocional representa um dos avanços mais promissores na interação humano-máquina. A capacidade de uma IA em reconhecer, interpretar e responder às emoções humanas pode revolucionar diversas áreas, desde a assistência pessoal até a saúde mental. Este relatório apresenta uma análise crítica das metodologias associadas à personalização e multimodalidade em IA emocional, baseando-se em estudos como \textcite{kargarandehkordi2024}, \textcite{gursesli2024} e \textcite{lee2024}. Além disso, integra-se a discussão com fundamentos teóricos de obras como \textcite{picard1997} e \textcite{pessoa2013}, bem como considerações éticas de \textcite{mueller2020}.
	
	Este trabalho visa não apenas avaliar os avanços tecnológicos, mas também refletir sobre suas implicações práticas e éticas, identificando desafios e propondo caminhos futuros para a pesquisa em IA emocional. A personalização de modelos afetivos e a análise multimodal são destacadas como abordagens chave para aumentar a precisão e a robustez dos sistemas de reconhecimento emocional, contudo, apresentam limitações que necessitam de atenção contínua.
	
	
	\section{Metodologia e Avanços}
	
	\subsection{Personalização de Modelos Afetivos}
	
	O artigo de \textcite{kargarandehkordi2024} foca na transição de abordagens generalistas para modelos personalizados na análise de emoções. Utilizando algoritmos clássicos de Machine Learning, como K-Nearest Neighbors e Random Forest, os autores demonstram que a personalização pode melhorar significativamente a precisão da classificação de emoções em cenários onde a variabilidade emocional intrapessoal é alta. Este avanço está alinhado com o conceito de adaptabilidade dos sistemas emocionais, discutido por \textcite{picard1997}, que argumenta que a compreensão das emoções humanas exige um modelo dinâmico capaz de lidar com variações contextuais e individuais.
	
	A personalização de modelos afetivos permite que sistemas de IA se ajustem às particularidades de cada usuário, proporcionando interações mais naturais e eficazes. No entanto, esse enfoque apresenta desafios significativos. Modelos personalizados, embora mais precisos, demandam uma quantidade considerável de dados específicos de cada usuário, o que pode ser inviável em larga escala. Além disso, a necessidade de treinar modelos com dados limitados pode introduzir vieses algorítmicos, comprometendo a equidade e a generalização dos sistemas (\textcite{mueller2020}).
	
	Adicionalmente, a implementação prática de modelos personalizados requer infraestrutura robusta para coleta, armazenamento e processamento de dados em tempo real. A escalabilidade desses modelos é uma preocupação central, especialmente em aplicações que envolvem um grande número de usuários com perfis emocionais diversos. Portanto, embora a personalização ofereça benefícios claros em termos de precisão, a viabilidade operacional e ética deve ser cuidadosamente considerada.
	
	
	\subsection{Análise Multimodal}
	
	No artigo de \textcite{gursesli2024}, o uso de redes neurais convolucionais leves (CNNs) para reconhecimento facial de emoções destaca-se como uma abordagem eficiente em termos de recursos computacionais. O modelo Custom Lightweight CNN Model (CLCM) desenvolvido pelos autores alcança desempenhos comparáveis a arquiteturas mais complexas, tornando-se uma opção viável para aplicações em tempo real onde a eficiência é crucial. Este avanço reflete a necessidade de equilibrar precisão e eficiência, especialmente em dispositivos com limitações de hardware.
	
	Por outro lado, o estudo de \textcite{lee2024} explora a análise multimodal, combinando dados de EEG, áudio e vídeo para reconhecer emoções em contextos conversacionais. Esta abordagem, alinhada ao levantamento de \textcite{poria2015}, demonstra que a integração de múltiplas modalidades de dados aumenta a robustez e a precisão dos sistemas de IA emocional. A combinação de diferentes tipos de dados permite capturar uma gama mais ampla de expressões emocionais, mitigando as limitações de métodos unimodais.
	
	Entretanto, a análise multimodal apresenta desafios técnicos significativos, especialmente na coleta e no processamento de dados. A sincronização e a integração de diferentes fontes de dados requerem algoritmos sofisticados e poder computacional elevado, o que pode limitar a aplicabilidade em ambientes com recursos restritos. Além disso, a privacidade e a segurança dos dados sensíveis, como sinais de EEG, tornam-se preocupações éticas que devem ser rigorosamente abordadas.
	
	Adicionalmente, a variabilidade interindividual nas expressões emocionais pode dificultar a generalização dos modelos multimodais. A diversidade cultural e contextual das expressões emocionais implica que os modelos devem ser treinados com datasets amplos e representativos para evitar vieses e garantir a eficácia em diferentes populações.
	
	
	\section{Integração de Emoção e Cognição}
	
	Os avanços descritos anteriormente refletem a interação complexa entre emoção e cognição no cérebro humano. \textcite{pessoa2013} argumenta que esses processos não podem ser dissociados, dado que sistemas emocionais e cognitivos interagem em redes cerebrais dinâmicas. Este entendimento é essencial para desenvolver sistemas de IA emocional que não apenas reconheçam emoções, mas também adaptem respostas de forma contextualizada e inteligente.
	
	Em "Emotion and Cognition", \textcite{pessoa2013} discute como as emoções influenciam a tomada de decisões e os processos cognitivos. Esse conhecimento é fundamental para a construção de IA que possa simular de maneira mais precisa as interações humanas. Por exemplo, um sistema de IA que reconhece a frustração em um usuário pode ajustar seu comportamento para oferecer suporte adicional ou modificar a interface para melhorar a experiência do usuário.
	
	No entanto, a integração bem-sucedida de emoção e cognição em sistemas de IA depende de uma compreensão profunda das heurísticas emocionais que influenciam as decisões humanas. \textcite{haidt2001} destaca que muitas decisões humanas são guiadas por intuições emocionais que precedem o raciocínio lógico. Ignorar essas interações pode levar a sistemas de IA que, embora tecnicamente sofisticados, falham em capturar a essência das interações humanas.
	
	Além disso, a integração de emoção e cognição em IA requer o desenvolvimento de modelos que possam aprender e adaptar-se continuamente a partir das interações com os usuários. Isso implica a necessidade de algoritmos que não apenas processam dados estáticos, mas que também incorporam feedback em tempo real para ajustar seus comportamentos de forma dinâmica.
	
	Portanto, a criação de sistemas de IA emocional eficazes e realistas exige uma abordagem interdisciplinar que combine insights da neurociência, psicologia e ciência da computação. A colaboração entre essas disciplinas pode levar ao desenvolvimento de modelos que replicam de maneira mais fiel a complexidade das emoções humanas e sua interação com os processos cognitivos.
	
	
	\section{Implicações Éticas e Sociais}
	
	O impacto ético do uso de IA emocional é um tema central discutido por \textcite{mueller2020}. Questões como privacidade, viés algorítmico e a possibilidade de manipulação emocional são preocupações crescentes que emergem com o avanço da IA emocional. A coleta de dados sensíveis, como expressões faciais e sinais de EEG, requer salvaguardas rigorosas para proteger os direitos e a privacidade dos usuários.
	
	Em contextos multimodais, como os apresentados por \textcite{lee2024}, o acesso a dados abrangentes aumenta a capacidade dos sistemas de IA emocional, mas também eleva os riscos associados à coleta e ao armazenamento desses dados. A utilização de dados biométricos e comportamentais pode levar a abusos se não forem implementadas políticas de proteção adequadas. Além disso, a dependência de dados específicos pode exacerbar desigualdades, especialmente se os sistemas forem treinados com representações limitadas de populações diversas.
	
	O viés algorítmico é uma preocupação significativa na personalização de modelos afetivos. Como \textcite{kargarandehkordi2024} apontam, a personalização pode introduzir vieses se os dados de treinamento não forem suficientemente representativos. Isso pode resultar em sistemas que funcionam melhor para determinados grupos de usuários, deixando outros sub-representados e potencialmente prejudicados.
	
	Por outro lado, \textcite{picard1997} argumenta que a IA emocional tem o potencial de melhorar significativamente a interação humano-máquina, desde que desenvolvida com responsabilidade. Sistemas personalizados podem adaptar-se às necessidades emocionais de utilizadores vulneráveis, como idosos ou pessoas com transtornos neurológicos, proporcionando suporte emocional e melhorando a qualidade de vida. No entanto, para realizar esse potencial, é crucial que esses avanços sejam acompanhados de regulamentações claras que garantam a segurança, a transparência e a ética no desenvolvimento e na implementação de tecnologias de IA emocional.
	
	Além disso, a possibilidade de manipulação emocional por meio de IA levanta questões éticas sobre a autonomia e o consentimento dos usuários. Sistemas de IA que influenciam ou alteram as emoções humanas podem ser utilizados de maneira abusiva, especialmente em contextos comerciais ou políticos. Portanto, é essencial que os desenvolvedores de IA emocional adotem princípios éticos rigorosos, garantindo que a tecnologia seja utilizada de forma a respeitar a dignidade e os direitos dos indivíduos.
	
	
	\section{Desafios e Oportunidades Futuras}
	
	A análise crítica dos artigos revisados revela que o campo da IA emocional está em uma encruzilhada entre inovação técnica e responsabilidade ética. Desafios como a escalabilidade de modelos personalizados, a coleta de dados multimodais e a mitigação de vieses precisam ser enfrentados para que os sistemas sejam amplamente adotados. Ao mesmo tempo, há oportunidades significativas para integrar esses avanços em áreas como saúde, educação e interação social.
	
	\subsection{Desafios}
	
	\begin{itemize}
		\item \textbf{Escalabilidade dos Modelos Personalizados:} A personalização de modelos afetivos, embora eficaz em aumentar a precisão, apresenta desafios em termos de escalabilidade. A necessidade de coletar e processar dados específicos para cada usuário pode limitar a aplicabilidade em cenários com um grande número de usuários.
		\item \textbf{Coleta e Processamento de Dados Multimodais:} A integração de múltiplas modalidades de dados, como EEG, áudio e vídeo, requer infraestrutura robusta e algoritmos avançados para sincronização e análise. Além disso, a coleta de dados sensíveis levanta preocupações éticas sobre privacidade e segurança.
		\item \textbf{Mitigação de Vieses Algorítmicos:} Garantir que os modelos de IA emocional sejam justos e representativos de diversas populações é um desafio contínuo. A falta de diversidade nos datasets de treinamento pode resultar em vieses que afetam a eficácia e a equidade dos sistemas.
		\item \textbf{Regulamentação e Ética:} Desenvolver frameworks éticos robustos que guiem o uso de IA emocional em contextos sensíveis é essencial para evitar abusos e proteger os direitos dos usuários.
	\end{itemize}
	
	\subsection{Oportunidades}
	
	\begin{itemize}
		\item \textbf{Integração em Saúde Mental:} Sistemas de IA emocional podem ser utilizados para monitorar e apoiar indivíduos com transtornos emocionais, proporcionando intervenções personalizadas e em tempo real.
		\item \textbf{Educação Personalizada:} A IA emocional pode adaptar métodos de ensino com base nas respostas emocionais dos alunos, melhorando a eficácia do aprendizado e aumentando o engajamento.
		\item \textbf{Interação Social Avançada:} Avanços em IA emocional podem levar ao desenvolvimento de assistentes virtuais e robôs sociais mais empáticos e capazes de interações mais naturais e significativas.
		\item \textbf{Pesquisa Interdisciplinar:} A colaboração entre neurociência, psicologia e ciência da computação pode impulsionar inovações que refletem com maior precisão a complexidade das emoções humanas.
	\end{itemize}
	
	\subsection{Caminhos de Investigação}
	
	Com base nos contributos dos artigos analisados, alguns caminhos de investigação e metodologias emergem como promissores:
	
	\begin{itemize}
		\item \textbf{Desenvolvimento de Frameworks Éticos:} Investigar e desenvolver frameworks que integrem princípios éticos no design e implementação de sistemas de IA emocional, garantindo transparência, responsabilidade e respeito aos direitos dos usuários.
		\item \textbf{Ampliar a Diversidade dos Datasets:} Coletar e utilizar datasets mais diversificados e representativos para treinar modelos de IA emocional, reduzindo vieses e melhorando a generalização dos sistemas.
		\item \textbf{Explorar Novas Modalidades de Dados:} Incorporar biomarcadores e outras fontes de dados fisiológicos que possam enriquecer a análise emocional sem comprometer a privacidade dos usuários.
		\item \textbf{Machine Learning Avançado:} Aplicar técnicas de aprendizado profundo e aprendizado por reforço para desenvolver modelos que possam aprender e adaptar-se continuamente a partir das interações com os usuários.
		\item \textbf{Avaliação de Impacto Social:} Conduzir estudos que avaliem o impacto social e psicológico dos sistemas de IA emocional, identificando benefícios e potenciais riscos associados à sua utilização.
	\end{itemize}
	
	
	\section{Conclusão}
	
	Este trabalho discutiu como a personalização e a análise multimodal representam avanços significativos no campo da IA emocional, ao mesmo tempo em que destacou lacunas e desafios éticos. A integração de emoção e cognição, como explorado por \textcite{pessoa2013} e \textcite{haidt2001}, é um passo crucial para desenvolver sistemas de IA que sejam tecnicamente eficazes e socialmente responsáveis. A personalização de modelos afetivos, conforme abordado por \textcite{kargarandehkordi2024}, e a análise multimodal, conforme demonstrado por \textcite{gursesli2024} e \textcite{lee2024}, mostram-se promissoras na melhoria da precisão e robustez dos sistemas de reconhecimento emocional.
	
	No entanto, os desafios relacionados à escalabilidade, coleta de dados sensíveis, mitigação de vieses e regulamentação ética não podem ser subestimados. É imperativo que a pesquisa futura não apenas se concentre em aprimorar as capacidades técnicas dos sistemas de IA emocional, mas também em desenvolver frameworks éticos robustos que guiem seu desenvolvimento e implementação responsável.
	
	Combinando inovação com reflexão ética, a IA emocional tem o potencial de transformar positivamente a interação humano-máquina, melhorando a qualidade de vida e promovendo interações mais empáticas e eficazes. Para alcançar esse potencial, é essencial que pesquisadores, desenvolvedores e legisladores trabalhem em conjunto para garantir que os avanços tecnológicos sejam acompanhados de considerações éticas e sociais apropriadas.
	
	\newpage
	
	% Referências
	\printbibliography
	
\end{document}
