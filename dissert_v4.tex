\documentclass[a4paper,12pt]{report}
\usepackage[utf8]{inputenc}
\usepackage[portuguese]{babel}
\usepackage{graphicx}
\usepackage{amsmath}
\usepackage{geometry}
\usepackage{setspace}
\usepackage{titlesec}
\usepackage{tocloft}
\usepackage{longtable}
\usepackage{lipsum}
\usepackage{enumitem}
\usepackage{fontspec}
\usepackage{xcolor}
\usepackage[hidelinks]{hyperref}
\usepackage{ragged2e}
\usepackage[natbibapa]{apacite} % Estilo APA para o natbib
\usepackage{url}

% Configuração da Margem e Espaçamento
\geometry{a4paper, margin=2.5cm}
\setstretch{1.5}

% Configuração de Fontes
\setmainfont{Times New Roman}

% Configuração de Títulos
\titleformat{\chapter}[block]{\normalfont\huge\bfseries}{\thechapter.}{20pt}{\centering}
\titleformat{\section}[block]{\normalfont\Large\bfseries}{\thesection}{1em}{}

% Configuração do Índice
\addto\captionsportuguese{%
	\renewcommand{\contentsname}{Índice} % Substitui "Conteúdo" por "Índice"
}
\renewcommand{\cftchapfont}{\bfseries}
\renewcommand{\cftsecfont}{\normalfont}
\renewcommand{\cftsubsecfont}{\itshape}
\renewcommand{\cftchappagefont}{\bfseries}
\renewcommand{\cftsecleader}{\cftdotfill{\cftdotsep}} % Ativa os pontos ligando títulos às páginas

% Configuração de Cores
\definecolor{barraazul}{RGB}{0, 51, 153}

% Começar numeração romana
\pagenumbering{roman}

\begin{document}
	
	% Capa
	\begin{titlepage}
		\begin{flushleft}
			\includegraphics[width=0.5\textwidth]{iscte.png}\\[1cm]
		\end{flushleft}
		\noindent
		\textcolor{barraazul}{\rule{\textwidth}{1mm}} % Barra azul
		\\[0.5cm]
		{\Huge \textbf{\centering Desenvolvimento de modelos preditivos com base em RNN}}\\[1.5cm]
		\noindent
		\textbf{Luís Ricardo Silva Inácio}\\
		\textbf{Número de Aluno: 129074}\\[2cm]
		\textbf{Mestrado em Inteligência Artificial}\\[1.5cm]
		\textbf{Orientador: Tozé Brito, Phd}\\
		\textbf{Coorientador: (se aplicável)}\\[3cm]
		\textbf{Outubro, 2024}
	\end{titlepage}
	
	% Página em Branco
	\newpage
	\thispagestyle{empty}
	\mbox{}
	\newpage
	
	% Contracapa
	\begin{titlepage}
		\begin{flushleft}
			\includegraphics[width=0.5\textwidth]{ista.png}\\[1cm]
		\end{flushleft}
		\noindent
		\textcolor{barraazul}{\rule{\textwidth}{1mm}} % Barra azul
		\\[0.5cm]
		{\Large \textbf{\centering Departamento de Ciências e Tecnologias da Informação}}\\[1cm]
		{\Huge \textbf{\centering Desenvolvimento de modelos preditivos com base em RNN}}\\[1.5cm]
		\noindent
		\textbf{Luís Ricardo Silva Inácio}\\
		\textbf{Número de Aluno: 129074}\\[2cm]
		\textbf{Orientador: Tozé Brito, Phd}\\
		\textbf{Coorientador: (se aplicável)}\\[3cm]
		\textbf{Outubro, 2024}
	\end{titlepage}
	
	% Página de Copyright
	\newpage
	\thispagestyle{empty}
	\noindent % Evita indentação na primeira linha
	{\footnotesize % Tamanho menor de texto (menor que \small)
		\textbf{Direitos de cópia ou Copyright} \\ % Título
		\textcopyright
		Copyright: Luís Ricardo Silva Inácio \\[-0.5cm] % Reduz o espaço após o autor
		\begin{flushleft} % Início do ambiente para alinhar à esquerda e justificar
			\renewcommand{\baselinestretch}{1}\selectfont % Define o espaçamento entre linhas
			\justify % Garante a justificação do texto
			O Iscte - Instituto Universitário de Lisboa tem o direito, perpétuo e sem limites geográficos, de arquivar e publicitar este trabalho através de exemplares impressos reproduzidos em papel ou de forma digital, ou por qualquer outro meio conhecido ou que venha a ser inventado, de o divulgar através de repositórios científicos e de admitir a sua cópia e distribuição com objetivos educacionais ou de investigação, não comerciais, desde que seja dado crédito ao autor e editor.
		\end{flushleft}
	}
	\newpage
	
	% Página de Agradecimentos
	\chapter*{Agradecimentos}
	\addcontentsline{toc}{chapter}{Agradecimentos}
	Gostaria de expressar a minha gratidão a todas as pessoas que me apoiaram durante a realização deste trabalho...
	
	% Resumo
	\newpage
	\chapter*{Resumo}
	\addcontentsline{toc}{chapter}{Resumo}
	Texto do resumo em português. \\[1em]
	\textbf{Palavras-chave:} palavra-chave1, palavra-chave2, palavra-chave3.
	\newpage
	
	% Abstract
	\chapter*{Abstract}
	\addcontentsline{toc}{chapter}{Abstract}
	Texto do resumo em inglês. \\[1em]
	\textbf{Keywords:} keyword1, keyword2, keyword3.
	\newpage
	
	% Índices
	\tableofcontents
	\newpage
	\listoffigures
	\newpage
	\listoftables
	\newpage
	
	% Lista de Abreviaturas e Siglas
	\chapter*{Lista de Abreviaturas e Siglas}
	\addcontentsline{toc}{chapter}{Lista de Abreviaturas e Siglas}
	\begin{longtable}{p{3cm} p{10cm}}
		\textbf{Sigla} & \textbf{Descrição} \\
		\hline
		API & Application Programming Interface \\
		BI & Business Intelligence \\ 
		KPI & Key Performance Indicator \\
	\end{longtable}
	
	% Começar numeração árabe
	\newpage
	\pagenumbering{arabic}
	
	% Introdução
	\chapter{Introdução}
	Texto fictício de introdução. 
	\lipsum[1-2]
	
	% Capítulos Seguintes
	\chapter{Revisão de Literatura}
	Texto fictício para revisão de literatura. 
	\lipsum[3-4]
	
	% Mock de Figuras
	\begin{figure}[h]
		\centering
		\includegraphics[width=0.5\textwidth]{example-image} % Use uma imagem real para substituir
		\caption{Exemplo de Figura 1}
		\label{fig:exemplo1}
	\end{figure}
	
	\chapter{Metodologia}
	Texto fictício para metodologia. 
	\lipsum[5-6]
	
	% Mock de Tabelas
	\begin{table}[h]
		\centering
		\begin{tabular}{|c|c|c|}
			\hline
			Coluna 1 & Coluna 2 & Coluna 3 \\
			\hline
			Dado 1 & Dado 2 & Dado 3 \\
			\hline
		\end{tabular}
		\caption{Exemplo de Tabela 1}
		\label{tab:exemplo1}
	\end{table}
	
	\chapter{Resultados}
	Texto fictício para resultados. 
	\lipsum[7-8]
	
	\chapter{Conclusão}
	Texto fictício para conclusão. 
	\lipsum[9]
	
	% Referências
	\chapter*{Referências Bibliográficas}
	\addcontentsline{toc}{chapter}{Referências}
	
	\begin{thebibliography}{}
		
		\bibitem{goodfellow2016} 
		Goodfellow, I., Bengio, Y., e Courville, A. (2016). *Deep Learning*. MIT Press.
		
		\bibitem{smith2021optimization} 
		Smith, J., Johnson, R., e Taylor, K. (2021). Optimization techniques for neural networks. *Journal of Machine Learning Research, 22*(3), 1–20. https://doi.org/10.1234/jmlr.v22i3.5678
		
		\bibitem{krizhevsky2012} 
		Krizhevsky, A., Sutskever, I., e Hinton, G. E. (2012). Imagenet classification with deep convolutional neural networks. *Advances in Neural Information Processing Systems, 25*, 1097–1105.
		
		\bibitem{brown2020} 
		Brown, P. (2020). Machine learning for predictive analytics. *Data Science Journal, 12*(4), 56–75.
		
		\bibitem{noauthor2021} 
		No Author. (2021). Recent advances in RNNs. *Neural Networks Review, 5*(1), 34–50.
		
		\bibitem{johnson1999as} 
		Johnson, L. (1999). As cited in Smith et al. (2021). *Introduction to AI*.
		
		\bibitem{wikipedia2024} 
		Wikipedia contributors. (2024). Recurrent neural networks. In *Wikipedia*. \url{https://en.wikipedia.org/wiki/Recurrent_neural_network}
		
		\bibitem{mitpress2018} 
		MIT Press. (2018). *Foundations of Deep Learning*. MIT Press.
		
		\bibitem{rao2019} 
		Rao, K. (2019). Challenges in neural networks. *AI Research Journal, 10*(2), 98–110.
		
		\bibitem{taylor2015} 
		Taylor, K. (2015). Simplifying neural networks for beginners. *Journal of AI, 8*(3), 25–35.
		
	\end{thebibliography}

	
	% Apêndices
	\appendix
	\chapter{Anexo A}
	Texto fictício para apêndice. 
	\lipsum[10]
	
\end{document}
