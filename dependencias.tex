\documentclass[a4paper,12pt]{report}

% Pacotes básicos
\usepackage[portuguese]{babel} % Idioma
\usepackage{graphicx} % Gráficos e imagens
\usepackage{amsmath} % Matemática avançada
\usepackage{geometry} % Margens e layout
\usepackage{setspace} % Espaçamento entre linhas
\usepackage{titlesec} % Personalização de títulos
\usepackage{tocloft} % Índice
\usepackage{longtable} % Tabelas longas
\usepackage{lipsum} % Texto fictício
\usepackage{enumitem} % Listas personalizáveis
\usepackage{fontspec} % Fontes personalizadas (necessário com XeLaTeX ou LuaLaTeX)
\usepackage{xcolor} % Cores
\usepackage[hidelinks]{hyperref} % Hiperlinks sem caixas
\usepackage{ragged2e} % Justificação de texto
\usepackage[natbibapa]{apacite} % Estilo APA para citações
\usepackage{url} % URLs
\usepackage{indentfirst} % Força a indentação do primeiro parágrafo

% Bibliografia
\bibliographystyle{apacite}

% Configuração de margens
\geometry{a4paper, top=2cm, bottom=2.5cm, left=2.5cm, right=2.5cm}

% Espaçamento entre linhas
\setstretch{1.5}

% Fontes
\setmainfont{Times New Roman}

% Configuração de títulos
% Configuração de títulos de capítulos alinhados à esquerda com tamanho 14pt
\titleformat{\chapter}[hang]
{\normalfont\bfseries\fontsize{14pt}{16pt}\selectfont} % Formatação do texto (14pt de tamanho, 16pt de espaçamento entre linhas)
{\thechapter.}{1em} % Número do capítulo seguido por um espaço de 1em
{\vspace{-1em}} % Reduz o espaço antes do título

% Configuração de subtítulos (sections) com tamanho 12pt
\titleformat{\section}[block]
{\normalfont\bfseries\fontsize{12pt}{14pt}\selectfont} % Formatação do texto (12pt de tamanho, 14pt de espaçamento entre linhas)
{\thesection}{1em}{}

% Índice e listas
\addto\captionsportuguese{%
	\renewcommand{\contentsname}{Índice} % Substitui "Conteúdo" por "Índice"
}
\renewcommand{\cftchapfont}{\bfseries}
\renewcommand{\cftsecfont}{\normalfont}
\renewcommand{\cftsubsecfont}{\itshape}
\renewcommand{\cftchappagefont}{\bfseries}
\renewcommand{\cftsecleader}{\cftdotfill{\cftdotsep}} % Ativa pontos ligando títulos às páginas

% Ajuste de cabeçalhos
\usepackage{fancyhdr}
\fancypagestyle{conteudo}{
	\fancyhf{} % Limpa cabeçalhos e rodapés
	\fancyhead[R]{\small \nouppercase{\leftmark}} % Título do capítulo no canto superior direito
	\renewcommand{\headrulewidth}{0pt} % Remove a linha no cabeçalho
}

% Reduz o espaçamento entre linhas dos índices
\usepackage{etoolbox}
\patchcmd{\tableofcontents}{\@starttoc{toc}}{\@starttoc{toc}\setlength{\parskip}{-5pt}}{}{}
\patchcmd{\listoffigures}{\@starttoc{lof}}{\@starttoc{lof}\setlength{\parskip}{-5pt}}{}{}
\patchcmd{\listoftables}{\@starttoc{lot}}{\@starttoc{lot}\setlength{\parskip}{-5pt}}{}{}

% Configuração de cores
\definecolor{barraazul}{RGB}{0, 51, 153}

% Corrigir problemas de viúvas e órfãs
\clubpenalty=10000
\widowpenalty=10000
\sloppy
